\chapter{Conclusion and outlook} % Main chapter title

\label{Chapter3} % Change X to a consecutive number; for referencing this chapter else where, use \ref{ChapterX}
%----------------------------------------------------------------------------------------
%	SECTION 1
%----------------------------------------------------------------------------------------

\section{Conclusion}

In the thesis we characterised spectrally the SiV$^{-}$ in nanodiamonds of 2 different size distributions with the help of confocal microscopy at cryogenic temperature. It has been noticed that the untreated SiV$^{-}$ in nanodiamonds are spectrally not stable, and SiV$^{-}$ smaller diamonds have worse spectral stability than larger ones. This spectral diffusion can be resolved in PL spectra. Previously it has been shown that the surface charging state plays a vital role in color centre luminescence. Negative charges on the surface can lead to depletion of NV \citep{stacey_depletion_2012}. For nanodiamonds, the band bending which originates from the difference of chemical potential in the bulk and on the surface is closely related to size-effect relating phenomena. It is highly suspected that the spectral diffusion of SiV$^{-}$ in nanodiamonds is also surface charge related.

Large number of PL spectra were taken to trace drift of lines. A standard method for spectral stability has been put forward. The calculation of mean cross-correlation can conclude the similarity of spectra into one number ranging from 0 to 1, Which makes statistical comparison possible. At the same time, time-resolved spectra realises the visualisation of spectra diffusion by plotting spectra over time in to colour maps.

Three surface treatment has been employed to modify the surface charging property. Aerobic Oxidation at 460$^{o}C$ - 480$^{o}C$ was used to selectively remove the graphitic defects on the surface while mild oxidation, resulting in carbonyl and carboxyl groups covered surface. Elevated temperature oxidation (575$^{o}C$) was used to initialize a bulk-diamond-like surface structure. Hydrogenation was to form a surface that posses negative electron affinity.

The oxidation at elevated temperature didn't produce useful data due to the distraction of substrate. The mildly oxidized sample showed heavily enhanced luminescence and worse spectral stability while the hydrogen terminated sample showed slightly reduced luminescence intensity and improved spectral stability.

%-----------------------------------
%	SECTION 2
%-----------------------------------
\section{ Outlook}

\paragraph{PL and PLE}
It is suspected that the ionization of Nitrogen atoms has been involved in the process, it is interested to compare the time-resolved spectra excited with a 532nm laser and a laser of 730nm.

PLE characterisation of the sample with improved spectra stability is also attractive, with reduced spectral diffusion, it might be finally possible for us to resolve the 4 line structure of SiV$^{-}$ ZPL.

\paragraph{life time measurement}
Due to the spectral diffusion, the orbital T1 measurement of untreated sample was not able to be carried out. With the improved spectral stability (need to be proved by PLE first), it might become possible. A longer time life will offers more possibility on the development of SiV$^{-}$ qubit.

\paragraph{Surface treatments}

Different surface treatments and comparison can help us understand the mechanism of spectral diffusion better. The most convenient treatments now are the hydroxylation which can be carried out directly on the hydrogenated sample resulting in a transform from negative electron affinity to positive electron affinity of the surface.

Anther interesting treatment is the dehydrogenation of sample by vacuum annealing, as the temperature varies, this can result in a clean diamond surface or a surface covered with thin-thick layer of graphite, which results in a gradual change in the surface electron affinity. \citep{diederich_electron_1998,maier_electron_2001}

Aerobic Oxidation still has more possibility to deal with. Since the size reduction rate of certain temperature is known, it is possible to profile the size effect on spectral stability by decreasing the size of nanodiamonds via oxidation gradually. It can also used to treat nanodiamonds that are too large (batch3, 4) for phonon elimination. The oxidation can also help to separate aggregated diamonds.

\paragraph{better method for size selection}
It is noticed, the size selection via centrifugation works, but is quite coarse. It might be possible to obtain finer batches by methods like high performance liquid chromatography. \citep{naoki_komatsu_chromatographic_2011}

\paragraph{relation between surface geometry and spectral behaviour}
Since it is possible to obtain the excitation polarisation pattern, it is interesting to observe the orientation preference of SiV$^{-}$/nanodiamonds statistically. This can be related to the formation process of HPHT diamond.
